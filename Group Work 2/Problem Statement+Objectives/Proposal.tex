\documentclass[12pt]{report}
\raggedright
\parindent0pt \parskip8pt
\usepackage{graphicx}

\begin{document}

\begin{Huge}
\begin{center}
\begin{normalsize}
\textbf{MAKERERE \includegraphics[scale=0.5]{logo} UNIVERSITY }\\


\textbf{FACULTY OF COMPUTING AND INFORMATICS TECHNOLOGY} \\
\textbf{SCHOOL OF COMPUTING AND INFORMATICS TECHNOLOGY} \\
\textbf{DEPARTMENT OF COMPUTER SCIENCE} \\
\textbf{BACHELOR OF SCIENCE IN COMPUTER SCIENCE} \\
\textbf{YEAR 2} \\
\textbf{BIT 2207 RESEARCH METHODOLOGY} \\
\textbf{Course Work: Assignment 5}\\
\end{normalsize}
\end{center}
\end{Huge}

\begin{center}
\begin{tabular}{|l|l|l|c|}
\hline NAME  & REG NO & STD NO \\\hline
OKOT EMMANUEL& 16/U/10916/PS & 216015844 \\\hline
MASIGA DAVID KELVIN& 16/U/579 & 216000507 \\\hline
NABWIIRE BABRA SANDRA& 16/U/10916/PS & 216011336 \\\hline
MULINDWA MATOVU KENNEDY& 16/U/10916/PS & 216021701 \\\hline
\end{tabular}
\paragraph{•}
GROUP F\\
\paragraph{•}
Lecturer: ERNEST MWEBAZE \\
\paragraph{•}
8th March 2018

\end{center}

\newpage

\title{SPEEDING DETECTOR ALONG ROADS TO CURB INCREASING ROAD TRAFFIC ACCIDENTS IN UGANDA IN VIEW OF TAXI DRIVERS}
\author{By Group F}      
\renewcommand{\today}{}

\maketitle
\tableofcontents

\chapter{Introduction}
\section{Background}
\paragraph{•}
Barely two months after 13 Tanzanian nationals perished on the Kampala-Masaka highway after attending a wedding in Kamoala, one Ugandan has died and three others were badly injured in an accident, on same road. Eye witnesses said the speeding FUSO truck lost control in a "dangerous spot" and swayed towards the side of the oncoming Saloon car, causing the crash at about 10:00pm. The car was damaged beyond repair. The driver of the Sallon car was travelling together with three passengers when this occurred. The three sustained serious injuries while the driver died on the scene.\cite{newvision} 
\paragraph{•}
Though the cause of the accident is still being investigated, road accidents have claimed the lives of many individuals, and is among the leading causes of injuries that are treated in a very odd way. Low income countries (LIC) and middle income countries (MIC) populations are faced with more road deaths than the high income countries (HIC) but all are as much affected.
\paragraph{•}
Regardless of what level of income an individual is, all road users are at risk of this endemic. The most hurting accidents are those that claim the life of children or pedestrians who are caught in the indiscipline of drivers to keep or maintain traffic rules. 
\paragraph{•}
Research has it that there are very many causes for road accidents; including - driver's age, weather, road conditions, driver's sex, drunkenness, attitudes of drivers and so on.\cite{hellen} Traffic experts agree that a good driving attitude-more than driving skill-can keep you out of most accident-producing situations from which not even the most skillful driver could escape without injury or property loss.\cite{stone} By observation, taxi drivers are well known for exhibiting a very bad attitude while driving. All these causes of accidents can be categorized as human error, environment factors and vehicle factors. In Uganda the process of tracking and  monitoring the behavior of drivers  has always been done  by the  traffic  officers  who often stand alongside roads and highways in specific areas using speed guns and radar antennas to catch over 
speeding  vehicles.  This  was  put  in  practice  after  sighting  over  speeding  as  one  of  the  main  causes  of  road accidents  by  the  traffic  police  of  Uganda  (Mubaraka,  2013).  However,  these  traffic  officers  cannot  be assembled in all spots of the roads to control the traffic. This has left the lives of very many Ugandans using road transport especially public transport at the mercy of irresponsible drivers.\cite{mukreport} The purpose of this research is to help improve the safety of passengers by designing a system that can be installed along blind spots to capture information on over speeding vehicles and report to the traffic police automatically.

\section{Lessons from Domestic and International Speed Camera Programs}
\paragraph{•}
Most researchers and public safety officials agree that speeding causes an increase in crashes. They generally agree that speed limit enforcement measures, including cameras, help catch and penalize drivers who break that speed limit. However, some questions remain unanswered. Do speed cameras actually lead to a reduction in the number of speeders and crashes, or reduce crash severity overall? \cite{boos}
\paragraph{•}
It has been an observation of many that when drivers are well acquainted with roads and visible speed cameras, as they approach an enforcement zone, they suddenly decelerate only to accelerate again after having passed it. There is no experimental proof that this leads to accidents however drivers will not have learned the detrimental results over speeding could lead to. In addition, some research has noted that drivers aware of fixed speed cameras may resort to using alternative routes to avoid the cameras, possibly leading to an increase in crashes on other roadways.
\paragraph{•}
Many drivers have actually discovered that not all speed cameras work, some are switched off. For this reason, speed cameras are actually ignored by such drivers.

\section{Problem Statement}
\paragraph{•}
Currently the Uganda traffic police use radar speed guns to detect over speeding vehicles. Traffic officers are deployed alongside roads, highways or streets to detect and stop over speeding vehicles. However, drivers exploit the loopholes within the current operational systems by reducing speed in areas where they suspect officers to be. This therefore does not reduce the over speeding and accident problems. The current traffic tracking system in Uganda has weaknesses which include the following;
\begin{itemize}
\item There are limited numbers of traffic officers therefore not enough to deploy on all roads in the entire country.
\item The cost of speed guns is high therefore this makes it hard to purchase them for every traffic officer. Consequently, just a few traffic officers possess speed guns. \cite{article4}
\end{itemize}
\paragraph{•}
While successful speed detector programs are in place in many countries today, the use of enforcement speed detectors is certainly contentious. Many international programs were initially met with public resistance. There are a number of public concerns transportation professionals may want to contemplate when considering or implementing a speed detector, including: general ticketing procedures, how ticket revenues will be distributed, privacy issues, and whether or not automated enforcement does result in reduced crash rates. Some studies have noted that public resistance to such programs can occur if speed detectors are perceived as revenue generators rather than methods for improving safety.
\paragraph{•}
In solving the problem of traffic violations along blind spots on different highways, the installation of speed detectors will be carried out in areas where traffic officers will be deployed and where they are not. On the upper hand, limitations in the number of traffic officers will be reduced. However, the cost to be incurred in the purchase of these devices will rise. 

\section{Objectives}
\subsection{General Objectives}
\end{document}